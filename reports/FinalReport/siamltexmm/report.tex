\documentclass[final]{siamltexmm}
\documentclass[10pt,a4paper]{article}

\usepackage{graphicx}
\usepackage{algorithm}
\usepackage{algorithmic}
\usepackage{mathtools}
\usepackage{amsmath}

% \usepackage[demo]{graphicx}
% \usepackage{subfig}

\newcommand{\pe}{\psi}
\def\d{\delta} 
\def\ds{\displaystyle} 
\def\e{{\epsilon}} 
\def\eb{\bar{\eta}}  
\def\enorm#1{\|#1\|_2} 
\def\Fp{F^\prime}  
\def\fishpack{{FISHPACK}} 
\def\fortran{{FORTRAN}} 
\def\gmres{{GMRES}} 
\def\gmresm{{\rm GMRES($m$)}} 
\def\Kc{{\cal K}} 
\def\norm#1{\|#1\|} 
\def\wb{{\bar w}} 
\def\zb{{\bar z}} 

% some definitions of bold math italics to make typing easier.
% They are used in the corollary.

\def\bfE{\mbox{\boldmath$E$}}
\def\bfG{\mbox{\boldmath$G$}}

\title{Independent Study -- Learning Music Structure by Laplacian Formula}
\author{Peter Yun-shao Sung\thanks{\tt yss265@nyu.edu} }

\begin{document}
\maketitle

\begin{abstract}
There are many approaches to analyzing music structure by features extracted from dimension of time series. With contruction of similarity matrix, repeated pattern can be captured which is the building block for large-scale structure. This is the work based on the Laplacian Matrix, which is essential start point of spectral clustering. We introduce variables that are trainable to reduce the cost of Laplacian Matrix from true lable, and run this method on wide variable of music recordings. Finally, we demonstrate using these trained variable for performing proper music segmentation.
\end{abstract}

\pagestyle{myheadings}
\thispagestyle{plain}

\section{Laplacian formula}
Normalized Laplacian matrix is the essential start point for identify music segmentation, and the correct boundary detection is done in my baseline approach [Ref2]

 formula as followed:
\begin{equation}
L := I - D^{1 \over 2}W D^{1 \over 2}
\end{equation}
D is degree matrix defined as the diagnal matrix with degrees $d_1, d_2, \ldots, d_n$, which $d_i$ is defined as followed:
\begin{equation}
d_i = \displaystyle\sum_{j \neq i}^{n} w_{ij}
\end{equation}
After multiplication and the result of equation 1.1 can be rewrite as:

\begin{equation}
L := I - D^{1 \over 2}W D^{1 \over 2} =
\begin{pmatrix}
  1 & {-w_{12} \over \sqrt{d_1d_2}} & \cdots & {-w_{1n} \over \sqrt{d_1d_n}} \\
  {-w_{21} \over \sqrt{d_2d_1}} & 1 & \cdots & {-w_{2n} \over \sqrt{d_2d_n}} \\
  \vdots  & \vdots  & \ddots & \vdots  \\
  {-w_{n1} \over \sqrt{d_nd_1}} & {-w_{n2} \over \sqrt{d_nd_2}} & \cdots & 1 \\
\end{pmatrix}
=
\begin{cases}
  {-w_{i,j} \over \sqrt{d_id_j}}       & \quad \text{if } i \neq j\\
  1   & \quad \text{if } i = j\\
\end{cases}
\end{equation}

for $w_{ij}$, where $i = j$:
\begin{equation}
{\partial L \over \partial w_{i,j}} =
\begin{cases}
  0       & \quad \text{, if $i = j$}\\
  {-1 \over \sqrt{d_id_j}} + {w_{i,j}(d_i+d_j) \over 2(d_id_i)^{3\over 2}} & \quad \text{, for position $(i,j), (j,i)$} \\
  {w_{l,k} \over 2\sqrt{d_k}d_l^{3\over2}}       & \quad \text{, for all position $(l,k), (k,l)$, where $k \neq i \& j$ and $l = i\|j$}\\
  0 & \quad \text{for any other position} \\
\end{cases}
\end{equation}

\begin{thebibliography}{10}
\bibitem{fpf} {\sc A Tutorial on Spectral Clustering}

\section{Appendix}
\subsection{Differential of Laplacian Matrix}
If we would like to take derivative of Laplacian w.r.t variable $w_{i,j}$ in the symmetric matrix $W$. Basically, except for $L_{i,j}$ the components of $L_{i,k}$, $L_{k,i}$, $L_{k,j}$, $L_{j,k}$ will need to consider.\\
For position $L_{i,j}$:
\begin{equation}
\begin{aligned}
L_{i,j} &= L_{j,i} =  {-w_{i,j} \over \sqrt{d_id_j}} \\
{\partial L_{i,j} \over \partial w_{i,j}} &= {\partial L_{j,i} \over \partial w_{i,j}} = {-1 \over \sqrt{d_id_j}} + {w_{i,j} \over 2(d_id_i)^{3\over 2}}({\partial d_i \over \partial w_{i,j}}d_j + d_i{\partial d_j \over \partial w_{i,j}}) \\
&= {-1 \over \sqrt{d_id_j}} + {w_{i,j}(d_i+d_j) \over 2(d_id_i)^{3\over 2}}
\end{aligned}
\end{equation}
For position $L_{l,k}$, $L_{k,l}$, where $k \neq i \& j$ and $l = i\|j$:
\begin{equation}
\begin{aligned}
L_{k,l} &= L_{l,k} =  {-w_{k,l} \over \sqrt{d_kd_l}} \\
{\partial L_{k,l} \over \partial w_{i,j}} &= {\partial L_{k,l} \over \partial w_{i,j}} = {w_{k,l} \over 2\sqrt{d_k}d_l^{3\over2}}
\end{aligned}
\end{equation}

\end{document}
