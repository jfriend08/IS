\documentclass[final]{siamltexmm}
\documentclass[10pt,a4paper]{article}

\usepackage{graphicx}
\usepackage{algorithm}
\usepackage{algorithmic}
\usepackage{mathtools}

% \usepackage[demo]{graphicx}
% \usepackage{subfig}

\newcommand{\pe}{\psi}
\def\d{\delta} 
\def\ds{\displaystyle} 
\def\e{{\epsilon}} 
\def\eb{\bar{\eta}}  
\def\enorm#1{\|#1\|_2} 
\def\Fp{F^\prime}  
\def\fishpack{{FISHPACK}} 
\def\fortran{{FORTRAN}} 
\def\gmres{{GMRES}} 
\def\gmresm{{\rm GMRES($m$)}} 
\def\Kc{{\cal K}} 
\def\norm#1{\|#1\|} 
\def\wb{{\bar w}} 
\def\zb{{\bar z}} 

% some definitions of bold math italics to make typing easier.
% They are used in the corollary.

\def\bfE{\mbox{\boldmath$E$}}
\def\bfG{\mbox{\boldmath$G$}}

\title{Independent Study -- Derivative of Laplacian formula}
\author{Peter Yun-shao Sung\thanks{\tt yss265@nyu.edu} }

\begin{document}
\maketitle

\pagestyle{myheadings}
\thispagestyle{plain}

\section{Laplacian formula}
Give the definition of normalized Laplacian formula as followed:
\begin{equation}
L := I - D^{1 \over 2}W D^{1 \over 2}
\end{equation}
D is degree matrix defined as the diagnal matrix with degrees $d_1, d_2, \ldots, d_n$, which $d_i$ is defined as followed:
\begin{equation}
d_i = \displaystyle\sum_{i \neq j}^{n} w_{ij}
\end{equation}
Althought the condition of $i \neq j$ is not emphasized in [1], but this is the condition when I checked scipy.sparse.csgraph.laplacian.\\
Here is the setps to calculate equation 1.1
\begin{equation}
D^{1 \over 2}W D^{1 \over 2} =
\begin{pmatrix}
  {1 \over \sqrt{d_1}} & 0 & \cdots & 0 \\
  0 & {1 \over \sqrt{d_2}} & \cdots & 0 \\
  \vdots  & \vdots  & \ddots & \vdots  \\
  0 & 0 & \cdots & {1 \over \sqrt{d_n}}
\end{pmatrix}
\begin{pmatrix}
  w_{11} & w_{12} & \cdots & w_{1n} \\
  w_{21} & w_{22} & \cdots & w_{2n} \\
  \vdots  & \vdots  & \ddots & \vdots  \\
  w_{n1} & w_{n2} & \cdots & w_{nn}
\end{pmatrix}
\begin{pmatrix}
  {1 \over \sqrt{d_1}} & 0 & \cdots & 0 \\
  0 & {1 \over \sqrt{d_2}} & \cdots & 0 \\
  \vdots  & \vdots  & \ddots & \vdots  \\
  0 & 0 & \cdots & {1 \over \sqrt{d_n}}
\end{pmatrix}
\end{equation}
After multiplication and the result of equation 1.1 can be rewrite as:
\begin{equation}
L := I - D^{1 \over 2}W D^{1 \over 2} =
\begin{pmatrix}
  1 - {w_{11} \over d_1} & {-w_{12} \over \sqrt{d_1d_2}} & \cdots & {-w_{1n} \over \sqrt{d_1d_n}} \\
  {-w_{21} \over \sqrt{d_2d_1}} & 1- {w_{22} \over d_2} & \cdots & {-w_{2n} \over \sqrt{d_2d_n}} \\
  \vdots  & \vdots  & \ddots & \vdots  \\
  {-w_{n1} \over \sqrt{d_nd_1}} & {-w_{n2} \over \sqrt{d_nd_2}} & \cdots & 1 - {w_{nn} \over d_n} \\
\end{pmatrix}
\end{equation}
and can be generalized as followed:
\begin{equation}
L_{i,j} =
\begin{cases}
  {-w_{i,j} \over \sqrt{d_i,d_j}}       & \quad \text{if } i \neq j\\
  1 - {w_{i,j} \over d_i}  & \quad \text{if } i = j \text{, if not-scipy}\\
  1   & \quad \text{if } i = j \text{, if not-scipy}\\
\end{cases}
\end{equation}


\begin{thebibliography}{10}
\bibitem{fpf} {\sc A Tutorial on Spectral Clustering}
\end{document}
